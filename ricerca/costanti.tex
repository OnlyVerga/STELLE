\textbf{\huge Tabella delle costanti}\\
\begin{table}[h!]
    \centering
    \begin{tabular}{c|c|c}
        \hline
        costante di gravitazione universale & G & $6,67 \cdot 10^{-11}\,\frac{N \cdot m^2}{kg^2}$\\
        massa della Terra & $m_t$ & $5,97 \cdot 10^{24}\,kg$\\
        massa di Marte & $m_m$ & $6,39 \cdot 10^{23}\,kg$ \\
        massa del Sole & $m_s$ & $ 1,99 \cdot 10^{30}\, kg$ \\
        accelerazione di gravità sulla superficie terrestre & $g_0$ & $9,8\,\frac{m}{s^2}$ \\
        raggio terrestre & $r_t$ & $6,37 \cdot 10^6 \, m$ \\
        raggio di Marte & $r_m$ & $3,40 \cdot 10^{6} \, m$ \\
        distanza Terra-Sole & $R_t$ & $1,50 \cdot 10^{11}\, m$ \\
        distanza Marte-Sole & $R_m$ & $2,28 \cdot 10^{11}\, m$ \\
        impulso specifico in atmosfera & $I_{spa}$ & $330\,s$ \\
        impulso specifico nel vuoto & $I_{sp}$ & $375\,s$ \\
        \hline
    \end{tabular}
    \caption{Tabella delle costanti}
    \label{tabella costanti}
\end{table}
\\
\textbf{\huge Sigle utilizzate}\\
\\
SOI: \textit{Sphere Of Influence}, indica la zona in cui il campo gravitazionale di un pianeta non è trascurabile.\\
\\
LEO: \textit{Low Earth Orbit}, letteralmente, orbita terrestre bassa, indica un insieme di orbite quasi circolari che partono da 300 km di altitudine fino a circa 1000 km.\\
\\
MECO: \textit{Main Engine CutOff}: letteralmente, spegnimento del motore principale. Di solito questa fase del volo coincide con il distaccamento del booster o del secondo stadio del lanciatore.\\
\\
SECO: \textit{Second Engine CutOff}: letteralmente, spegnimento del secondo motore. Una volta avvenuto il SECO il razzo si trova in orbita e si chiude la fase di lancio per dare inizio alla fase di controllo delle apparecchiature.