L'atterraggio di un'astronave Starship è un concetto nuovo nel campo dell'ingegneria aerospaziale, in quanto nessuno ci aveva mai provato prima. La stessa SpaceX, più di 10 anni prima, aveva progettato il Falcon, un lanciatore il cui primo stadio sarebbe stato riutilizzabile. Quel progetto si è evoluto fino a diventare il Falcon 9 che oggi conosciamo.\\
Con Starship, anche se il concetto di base è lo stesso, si porta la riutilizzabilità a tutto un altro livello: il booster Super Heavy atterrerà in un modo simile al primo stadio del Falcon, con la differenza che verrà "acchiappato" da una torre al posto di eseguire un atterraggio completo.\\
La vera differenza sta nel fatto che se il Falcon è pensato per portare dei payload nello spazio la Starship è pensata per lo scopo di trasportare umani, quindi è necessario che anche il secondo stadio sia riutilizzabile. Per ottenere questo risultato gli ingegneri di SpaceX hanno inventato un metodo di atterraggio in planata simile agli Space Shuttle della NASA, con la sostanziale differenza che Starship, una volta effettuato il rientro in atmosfera, si riposizionerà in verticale.\\
Questo metodo è estremamente efficace se si parla di consumi, in quanto i motori devono essere accesi solo nella parte finale del rientro. \\
L'atterraggio su Marte risulta essere ancora più semplice in quanto la gravità è minore.\\
Starship si basa su un sistema a metano e ossigeno liquido, entrambi facilmente estraibili dal suolo marziano. I coloni potranno quindi rifornire la Starship intanto che vivranno su Marte e ripartire alla successiva finestra di lancio.\\
Per partire da Marte non c'è neanche bisogno di un booster Super Heavy in quanto la gravità è molto minore di quella sulla Terra.\\
Le potenzialità di questo sistema sono innumerevoli, dal lancio di satelliti in bassa orbita terrestre a satelliti per le telecomunicazioni in orbita geostazionaria, fino a viaggi verso Marte e oltre.\\
\\
\\
\\