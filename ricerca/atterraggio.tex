\subsection{Atterraggio}
L'atterraggio di un'astronave Starship è un concetto nuovo nel campo dell'ingegneria aerospaziale, in quanto nessuno ci aveva mai provato prima. La stessa SpaceX, più di 10 anni prima, aveva progettato il Falcon, un lanciatore con la possibilità di riutilizzare il primo stadio per lanci futuri. Quel progetto si è evoluto fino a diventare il Falcon 9 che oggi conosciamo.\\
Con Starship, anche se il concetto di base è lo stesso, si porta la riutilizzabilità a tutto un altro livello: il booster Super Heavy atterrerà in un modo simile al primo stadio del Falcon, con la differenza che verrà "acchiappato" da una torre al posto di eseguire un atterraggio completo.\\
Questo poi verrà rifornito di carburante e sarà pronto per ripartire.\\
La vera differenza sta nel fatto che se il Falcon è pensato per portare dei payload nello spazio la Starship è pensata per lo scopo di trasportare umani, quindi è necessario che anche il secondo stadio sia riutilizzabile. Per ottenere questo risultato gli ingegneri di SpaceX hanno inventato un metodo di atterraggio in planata simile agli Space Shuttle della NASA, con la sostanziale differenza che Starship, una volta effettuato il rientro in atmosfera, si riposizionerà in verticale.\\
Questo metodo è estremamente efficace se si parla di consumi, in quanto i motori devono essere accesi solo nella parte finale del rientro. \\
L'atterraggio su Marte risulta essere ancora più semplice in quanto la gravità è minore.\\
Queste innovazioni sono rivoluzionarie nell'industria aerospaziale in quanto permettono di effettuare molti lanci in un intervallo limitato di tempo.\\
Con un numero relativamente piccolo di Super Heavy sarà possibile lanciare un numero importante di Starship componendo una vera e propria flotta con destinazione Marte.
\subsection{Refueling e ripartenza}
Il sistema Starship si basa su metano e ossigeno liquido per alimentare i motori. Questa scelta non è casuale perché entrambi i composti sono comuni sulla superficie di Marte.\\
Insieme ai coloni verranno mandati dei robot scavatori con il compito di minare il carburante necessario alla ripartenza.\\
Durante i primi lanci Starship con destinazione Marte verranno testate queste tecnologie per assicurarsi che funzionino al meglio.\\
Una volta ottenuto il carburante necessario e arrivata una nuova finestra di lancio i coloni saranno pronti a ripartire verso la Terra. La gravità di Marte è circa il 38\% di quella terrestre, il che aiuta non poco se si prende in considerazione il fatto che per ripartire la Starship non è dotata di booster Super Heavy.\\
    