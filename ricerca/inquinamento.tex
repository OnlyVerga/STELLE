\subsection{Resoconto dei consumi}
Per concludere questa prima parte non rimane altro che calcolare effettivamente quanto inquinamento produce questo lancio.\\
È possibile trascurare il carburante speso per far atterrare il booster o la Starship poiché è minimo in quanto SpaceX sta progettando un sistema di atterraggio in planata che porta ad accendere i motori solo agli ultimi attimi prima dell'atterraggio.\\
Inoltre i calcoli fatti in precedenza sono validi anche per la Starship "di supporto" in quanto si riferiscono ad un'astronave a pieno carico. In questo caso sarebbe riempita di carburante.\\
Per riassumere ecco i principali momenti di accensione del motore:\\
\\
\begin{table}[h!]
\centering
\label{tablella consumi}

\begin{tabular}{|l|l|}
    \hline
    lancio & 3\,400 tonnellate in bassa atmosfera e 937 tonnellate in alta atmosfera  \\
    lancio di una Starship per il \textit{refueling} & 3\,400 tonnellate in bassa atmosfera  \\
    inserimento in orbita iperbolica & 820 tonnellate \\
    \hline
    totale & 8\,557 tonnellate \\
    \hline
    
\end{tabular}
\caption{riassunto di tutti i maggiori consumi durante una missione}
\end{table}\\
I consumi totali ammontano quindi a 8\,557 tonnellate di carburante.\\ Di questi solo una parte contribuiscono all'inquinamento atmosferico e al riscaldamento globale, l'altra parte possono essere gas innocui o possono disperdersi nelle regioni più alte dell'atmosfera.\\
Abbiamo ignorato eventuali correzioni orbitali necessarie durante il viaggio, in quanto essendo lontani dalla Terra non possono causare alcun danno.\\
Inoltre, in quanto il nostro obiettivo è focalizzarci sull'inquinamento terrestre, abbiamo tralasciato anche i consumi in orbita marziana.
\subsection{Calcolo dei gas serra prodotti}
La reazione che accade nel motore raptor è:
$$
\ce{CH4} + \ce{O2} \rightarrow \ce{CO2} + \ce{H2O}
$$
e bla bla bla cose da chimici