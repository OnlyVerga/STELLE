\textbf{\large Le auto inquinano. E questo è ormai risaputo. Ma quanto inquinano?}\\

È capitato sicuramente a tutti di sentire che è ora di smetterla di usare automobili inquinanti e che bisogna passare a modelli più puliti. Si sente dappertutto parlare dei cambiamenti climatici, di gas serra, foreste e molto altro. Ci si preoccupa ogniqualvolta al telegiornale si sente parlare di disastri meteorologici, ma spesso non si ha né la voglia, né il tempo di provare a cercare cosa significhi veramente inquinare. Provando a fare qualche ricerca invece si scopre quanto sia urgente una svolta. Nella tabella \ref{tabella auto} si espone qualche dato sulle auto più vendute negli ultimi anni al mondo, auto che, pur non essendo spesso presenti nello scenario italiano, sono nelle classifiche mondiali. Premesso che, ovviamente, i valori di CO2 emessi variano persino sul tipo di tratto percorso, la versione, etc...
\begin{table}[h!]
    \centering
    \begin{tabular}{|c|c|c|}
    \hline
        \textbf{\Large Autovettura} & \textbf{\Large Consumi (km/L)} & \textbf{\Large Emissioni di CO2 (g/km)} \\
        \hline
        Toyota Corolla & 20-24* & 167,9 \\
        \hline
        Volkswagen Polo & 18,87 & 136,9 \\
        \hline
        Honda Civic Type R & 12,99 & 152,6 \\
        \hline
        Volkswagen Tiguan 1.5 TSI ACT R-Line DSG & 13,89 & 166,2 \\
        \hline
        Honda Cr-v & 13,51 & 188,5 \\
        \hline
    \end{tabular}
    \caption{Consumi di alcuni modelli di automobili più famosi}
    \label{tabella auto}
\end{table}
\small *CONSUMI ED EMISSIONI WLTP (ai sensi del Regolamento UE 2017/1151) 1.8 HYBRID - Dati riferiti alla versione con cerchi da 16" e da 17"

Nella seconda colonna vengono presentati i consumi delle auto in km/L; questo naturalmente significa che i valori più alti sono ben accetti, visto che l’autovettura riesce in questo modo a percorrere più chilometri avendo a disposizione solo un litro di benzina. E si può notare con piacere che la Toyota Corolla, l’auto più venduta nel mondo da qualche anno a questa parte, riesce a percorrere un tragitto più lungo rispetto alla media delle altre automobili. Probabilmente questo è uno dei fattori che ha contribuito a farla diventare la più comprata. Guadagni a parte, questo significa che l’auto più venduta al mondo spreca meno benzina. L’automobile peggiore in questa piccola classifica sembra essere Honda Civic, che necessita quindi di più benzina di tutte le altre. 
Nella terza colonna avviene l’esatto contrario; ovvero più piccoli sono i valori meglio è. Come si può notare dalla tabella questi valori sono ancora molto alti, e questo è un problema. Per quanto si tenga conto che le fonti segnalino valori molto diversi a seconda di troppi fattori, è preoccupante comunque notare che nessuna tra queste auto scende sotto i 100 g/km. Se pensiamo che la Volkswagen Polo è quella che presenta il valore più basso tra queste automobili, c’è di che riflettere. Ogni volta che qualcuno sale su una Polo e percorre un chilometro, libera quasi 137 g di CO2. Pensando a quanti chilometri si coprono ogni giorno, a quanti chilometri vengono percorsi da auto di tutto il pianeta, ne esce fuori una quantità enorme di CO2 che finisce nell’atmosfera ogni giorno. Ogni giorno, per tutta la settimana, per tutto l’anno. E così via all’infinito. 
A questo punto viene da chiedersi: ma quindi, quanta CO2 viene emessa ogni anno? La domanda è molto interessante ma non è di facile risposta. Come sempre i fattori sono molto diversi e tutto questo porta a delle differenze evidenti. Comunque “stando alle statistiche dell'ICDP – International Car Distribution Program - in Italia, gli automobilisti percorrono circa 12.000 Km. Sebbene il dato sia aggiornato soltanto al 2015 e manchino dei dati di confronto più recenti, si tratta comunque di una situazione interessante: l'istituto di ricerca, infatti, durante il sondaggio precedente, aveva previsto un calo dell'utilizzo del veicolo privato di almeno 1.000 km l'anno ma così non è stato. I guidatori, infatti, continuano a preferire la propria auto per recarsi al lavoro o per viaggiare e sembra che molti siano ancora restii a preferire i mezzi pubblici o dei sistemi di trasporto alternativi che possano ridurre il traffico nelle strade, oltre alle emissioni di gas nocivi. La media indica anche le zone di maggiore utilizzo dei mezzi privati: i guidatori del sud Italia sembrano decisi a voler utilizzare l'auto anche per recarsi al lavoro o per svolgere i piccoli doveri quotidiani, mentre al nord si preferisce utilizzare il tram o il bus, anche per evitare lo stress causato dal dover trovare un parcheggio per il proprio veicolo.”  Dodicimila chilometri all’anno. Non trovando, purtroppo, un dato che sia globale, ci accontenteremo del dato italiano, usando a modello l’auto Fiat Panda, quella più comprata dagli italiani. Sapendo che la Panda emette 133 g/km 3 di CO2 e tenendo conto che in effetti si tratta di un valore piuttosto basso rispetto alle 5 auto della tabella di sopra, si svolge il calcolo. Ne risulta che in media ogni italiano, supponendo che usi una Fiat Panda a benzina, emette 1.596.000 grammi di CO2, ovvero 1596 kg.  Praticamente il peso di un grosso tricheco. Ogni anno un italiano, andando in auto, mette un tricheco di CO2 in più nell’atmosfera. E in Italia, in tutto, siccome secondo l’Istat 3ultimo le auto in circolazione nel 2021 sono 39 545 232 (più di 39 milioni) vengono emessi $6,311*10^{10}$ kg di CO2. Dunque più di $60\,000\,000\,000$ kg. Una cifra spaventosa ed enormemente lunga. Ma non è finita.

L’anidride carbonica non è l’unico gas serra emesso dalle auto, anche se è il più importante. La Toyota Corolla, da qualche anno l’auto più venduta al mondo, nella scheda tecnica3ultimo segna infatti altre emissioni, secondo la Direttiva 1999/100/EC.
\newpage